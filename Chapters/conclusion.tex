% Chapter Template

\chapter{Conclusions} % Main chapter title

\label{Conclusion} % Change X to a consecutive number; for referencing this chapter elsewhere, use \ref{ChapterX}

\lhead{Conclusion. \emph{Conclusion}} % Change X to a consecutive number; this is for the header on each page - perhaps a shortened title


Always in every Engineering systems there is trade off which Communication Systems also is not an exception of this history, each day we are going to develop our ideas about something maybe a great technological idea like IoT, SDN, ... . Actually when you are an engineer that you understood when it should use a clever way to solve your problem. 'genius' is to apply a science in the real world is just a combination of art and science and it's beautiful.

To be a good engineer we should learn all of tools that are going to be useful and they define your power to progress your work without problems. Some days you don't get a good result, it's a \textit{good news }because you got a great experience to not repeat it and it must always try to find the answers and suitable results. So it seems that you should have a project and you must define a goal for your project.

Network protocols are completely different by several reasons and visions so depends on your system your designing must be varied. In Wireless and Wired Networks we are limited in some constraints who are mainly definitive for our objects, control of objects from go away is now our ideas because the thing that connect our real world to Computer, Internet or generally virtual worlds are \textit{Sensors}

Computer Science is the science that i can implement and see my results and that's it's beautiful. when you understand how it works.


Routing algorithms should exist in the routers and how this is perfect. The very basic idea is that each router should know what should it does with packets. This logic which should be implemented is the core of one network. Because otherwise you can not flow your packets to the good destinations. Our strategies are proposed to solve this fundamental question and need.
     



%----------------------------------------------------------------------------------------
%	SECTION 1
%----------------------------------------------------------------------------------------

%\section{Internship Experience Summary}
%
%Lorem ipsum dolor sit amet, consectetur adipiscing elit. Aliquam ultricies lacinia euismod. Nam tempus risus in dolor rhoncus in interdum enim tincidunt. Donec vel nunc neque. In condimentum ullamcorper quam non consequat. Fusce sagittis tempor feugiat. Fusce magna erat, molestie eu convallis ut, tempus sed arcu. Quisque molestie, ante a tincidunt ullamcorper, sapien enim dignissim lacus, in semper nibh erat lobortis purus. Integer dapibus ligula ac risus convallis pellentesque.
%
%\section{Main Curves and Results of Internship}

%\subsection{Recommendations for Company}


%-----------------------------------
%	SUBSECTION 1
%-----------------------------------
%\subsection{Recommendations for University}
%
%Nunc posuere quam at lectus tristique eu ultrices augue venenatis. Vestibulum ante ipsum primis in faucibus orci luctus et ultrices posuere cubilia Curae; Aliquam erat volutpat. Vivamus sodales tortor eget quam adipiscing in vulputate ante ullamcorper. Sed eros ante, lacinia et sollicitudin et, aliquam sit amet augue. In hac habitasse platea dictumst.

%-----------------------------------
%	SUBSECTION 2
%-----------------------------------
%
%\subsection{Algorithms of Encoding and Results}
%Morbi rutrum odio eget arcu adipiscing sodales. Aenean et purus a est pulvinar pellentesque. Cras in elit neque, quis varius elit. Phasellus fringilla, nibh eu tempus venenatis, dolor elit posuere quam, quis adipiscing urna leo nec orci. Sed nec nulla auctor odio aliquet consequat. Ut nec nulla in ante ullamcorper aliquam at sed dolor. Phasellus fermentum magna in augue gravida cursus. Cras sed pretium lorem. Pellentesque eget ornare odio. Proin accumsan, massa viverra cursus pharetra, ipsum nisi lobortis velit, a malesuada dolor lorem eu neque.
%
%%----------------------------------------------------------------------------------------
%%	SECTION 2
%%----------------------------------------------------------------------------------------
%
%\subsection{Algorithms of Decoding and Results}
%
%Sed ullamcorper quam eu nisl interdum at interdum enim egestas. Aliquam placerat justo sed lectus lobortis ut porta nisl porttitor. Vestibulum mi dolor, lacinia molestie gravida at, tempus vitae ligula. Donec eget quam sapien, in viverra eros. Donec pellentesque justo a massa fringilla non vestibulum metus vestibulum. Vestibulum in orci quis felis tempor lacinia. Vivamus ornare ultrices facilisis. Ut hendrerit volutpat vulputate. Morbi condimentum venenatis augue, id porta ipsum vulputate in. Curabitur luctus tempus justo. Vestibulum risus lectus, adipiscing nec condimentum quis, condimentum nec nisl. Aliquam dictum sagittis velit sed iaculis. Morbi tristique augue sit amet nulla pulvinar id facilisis ligula mollis. Nam elit libero, tincidunt ut aliquam at, molestie in quam. Aenean rhoncus vehicula hendrerit.
%
%\section{New Channel Coding (Reed-Solomon) }
%
%\subsection{Algorithms of Encoding and Results}
%
%
%\subsection{Algorithms of Decoding and Results}
%
%
%\section{Conclusion and Results}