\documentclass[8pt]{beamer} 
%\usepackage{pgf}
\usepackage{animate}
\usepackage{graphicx}
\usepackage{subfigure}

\usepackage[utf8]{inputenc}
\usepackage[T1]{fontenc}
\usepackage{lmodern}

\usepackage{amssymb,amsmath,amscd,amsfonts,amsthm,bbm,bm}
\usepackage{wasysym}
%\usepackage{graphicx}
%\usepackage{animate}
%\usepackage{psfrag}
%\usepackage{paralist}
\usepackage{color}
%\usepackage{float}
%\usepackage{relsize}
%\usepackage{euscript}
%\usepackage{setspace}
%\usepackage{flushend}
%%\usepackage[usenames,dvipsnames]{xcolor}
\usepackage{tikz}

\usepackage{pgfplots}
\usetikzlibrary{arrows,shapes,decorations,backgrounds}
\usepackage{tkz-berge}
\DeclareMathOperator*{\argmin}{argmin}
\DeclareMathOperator*{\argmax}{argmax}
\newcommand{\SNR}{\mathsf{SNR}}
\newcommand{\Q}{\mathbb{Q}}
% *** TPT STYLE ***

\setbeamerfont{subsection in toc}{size=\large}

\definecolor{colorframetitle}{RGB}{191,18,56} % Title of frames
\definecolor{redbox}{RGB}{191,18,56} % 
\definecolor{blackbox}{RGB}{0,0,0} % 
\definecolor{brownbox}{RGB}{128,99,90} % 

\usepackage[absolute,overlay]{textpos}
\usepackage{listings}
\usepackage{hyperref}

\setlength{\TPHorizModule}{1mm}
\setlength{\TPVertModule}{1mm}

\usepackage{tikz}
\usetikzlibrary{decorations.pathreplacing,calc}
\usetikzlibrary{arrows,shapes,snakes,automata,backgrounds,petri}
\usepackage{scalefnt}



\newcommand{\tikzmark}[1]{\tikz[overlay,remember picture] \node (#1) {};}

\tikzstyle{mybox} = [draw=redbox, fill=redbox!20, very thick,
    rectangle, rounded corners, inner sep=10pt, inner ysep=20pt]
\tikzstyle{fancytitle} =[fill=redbox, text=white, rectangle]


\newcommand{\MyLogo}{
%\begin{textblock}{14}(117.2,0.7)
\begin{textblock}{14}(108.8,86)
\includegraphics[width=0.8cm]{figures/tpt}
\includegraphics[width=1.1cm]{figures/cisco.png}
 \end{textblock}
}


%\begin{textblock}{14}(117.2,0.7)








\usepackage{beamerthemesplit}

\setbeamercolor{itemize item}{fg=redbox}
\setbeamercolor{structure}{fg=redbox, bg=red}
\setbeamercolor{block title}{bg=brownbox,fg=white}
\setbeamercolor{block title alerted}{bg=redbox,fg=white}
\setbeamercolor{block body alerted}{bg=brownbox!0,fg=black}
\setbeamercolor{block title example}{bg=black, fg=white}
\setbeamercolor{palette primary}{fg=black,bg=white} % changed this
\setbeamercolor{palette secondary}{use=structure,fg=structure.fg!100!white} % changed this
\setbeamercolor{palette tertiary}{use=structure,fg=structure.fg!100!white} % changed this
\setbeamercolor*{palette quaternary}{fg=black,bg=white} % outline on top left
\setbeamercolor{background canvas}{bg=white, fg=black} 
\setbeamercolor{frametitle}{fg=colorframetitle}


% First  frame
\newcommand{\RectanglesOfMainSlide}{%
\raisebox{0mm}[0pt][0pt]{%
\begin{pgfpicture}{0mm}{0mm}{0mm}{0mm}
\pgfsetlinewidth{5mm}
\color{redbox}
\pgfline{\pgfpoint{-4mm}{-12mm}}{\pgfpoint{24mm}{-12mm}}
\color{blackbox}
\pgfline{\pgfpoint{24mm}{-12mm}}{\pgfpoint{52mm}{-12mm}}
\color{brownbox}
\pgfline{\pgfpoint{52mm}{-12mm}}{\pgfpoint{80mm}{-12mm}}
\end{pgfpicture}}}

\newcommand{\makeFirstFrame}{
\setbeamertemplate{footline}{} 
\frame[plain]{
\begin{columns}[c]
\column{3cm}
\vspace{-2cm}\\
\includegraphics[width=1.5cm]{figures/tpt}
\includegraphics[width=2.2cm]{figures/cisco.png}\\
\column{7cm}
\vspace{1cm}\\
\LARGE{\textbf{\theTitle}}\\
\vspace{0.5cm}
\normalsize{\theAuthors}\\
\vspace{0.5cm}
\normalsize{\theResponsible}\\
\vspace{0.5cm}

\normalsize{\theConferenceAndPlace}\\
\vspace{-1cm}
\RectanglesOfMainSlide
\end{columns}
}
\activateFootline
}


% Frames decoration

\newcommand{\RectanglesBeforeTitle}{%
\raisebox{0mm}[0pt][0pt]{%
\begin{pgfpicture}{0mm}{0mm}{0mm}{0mm}
\pgfsetlinewidth{5mm}
\color{redbox}
\pgfline{\pgfpoint{-2mm}{2.2mm}}{\pgfpoint{4mm}{2.2mm}}
\color{blackbox}
\pgfline{\pgfpoint{4mm}{2.2mm}}{\pgfpoint{10mm}{2.2mm}}
\color{brownbox}
\pgfline{\pgfpoint{10mm}{2.2mm}}{\pgfpoint{16mm}{2.2mm}}
\end{pgfpicture}}}

\setbeamertemplate{frametitle}{
\begin{columns}[t]
\column{16mm}
\RectanglesBeforeTitle 
\column{10.7cm}
\strut\textbf{\insertframetitle}\strut
\end{columns}
}

% Foot line

\newcommand{\Ffootline}{
\MyLogo
\begin{tikzpicture}
 \fill [color=white, fill=redbox] (-1, -0.05) rectangle (1, 0.30);
\node[white, right] (note1) at (-1, 0.10) {\insertframenumber/\inserttotalframenumber};
\node[white, left] (note1bis) at (0.98, 0.10) {\theDate};
 \fill [color=white, fill=blackbox] (1.05, -0.05) rectangle (4.5, 0.30);
\node[white, align=center] (note2) at (2.27, 0.12) {Institut Mines-Telecom};
 \fill [color=red, fill=brownbox] (3.55, -0.05) rectangle (9.75, 0.30);
\node[white, align = center] (note3) at (6.65, 0.10) {\thefootnotedef};

%\node[white] (note3) at (7.5, 0.10) {\theTitle};
\end{tikzpicture}
}

\newcommand{\activateFootline}{
\setbeamertemplate{footline}{
\usebeamerfont{structure}
\Ffootline
}
}


% *** END OF TPT STYLE ***


%remove navigation symbols
\setbeamertemplate{navigation symbols}{}

% To show the outline at the beginning of each section
\AtBeginSection[]{
   \begin{frame}
   \frametitle{Plan}
   %\begin{center}{\LARGE Outline }\end{center}
   \tableofcontents[currentsection,hideothersubsections]
   \end{frame} 
}

\newcommand{\mytilde}{\raise.17ex\hbox{$\scriptstyle\mathtt{\sim}$}}

\newcommand{\tikzgrid}{
\begin{pgfonlayer}{background}
\draw[gray!50]
(current bounding box.south west)
grid[step=.2] (current bounding box.north east);
\draw[red!50]
(current bounding box.south west)
grid (current bounding box.north east);
\end{pgfonlayer}
}

\newcommand*{\ExtractCoordinate}[3]{\path (#1); \pgfgetlastxy{#2}{#3};}%

\newdimen\tlx
\newdimen\tlx
\newdimen\brx
\newdimen\bry

%% To FILL to customize presentation with the TPT style

\newcommand{\theTitle}{Routing Algorithms in NDN Networks}
\newcommand{\thefootnotedef}{Routing Algorithms in NDN networks}
\newcommand{\theAuthors}{shahab SHARIAT BAGHERI}
\newcommand{\theResponsible}{Luca MUSCARIELLO\\Pablo PIANTANIDA \\ Beatrice PESQUET \\ Jean Le Feuvre}
\newcommand{\theConferenceAndPlace}{Internship Defense \\
Salle F801, TELECOM ParisTech \\ 10:00 AM, 9/19/2016}
\newcommand{\theDate}{9/19/2016}

%%%%%%

\DeclareMathOperator{\tr}{Tr}
\DeclareMathOperator{\support}{supp}
\DeclareMathOperator{\rank}{rank}
\DeclareMathOperator{\diag}{diag}
\newcommand{\bs}{\boldsymbol}

% Convergences 
\newcommand{\toaslong}{\xrightarrow[T\to\infty]{\text{p.s.}}}
\newcommand{\toasshort}{\xrightarrow{\text{p.s.}}}
\newcommand{\toprobalong}{\xrightarrow[T\to\infty]{{\cal P}}}
\newcommand{\toprobashort}{\xrightarrow{{\mathcal P}}}
\newcommand{\tolawlong}{\xrightarrow[T\to\infty]{{\cal L}}}
\newcommand{\tolawshort}{\xrightarrow{{\mathcal L}}}
\newcommand{\tolong}{\xrightarrow[T\to\infty]{}}
\newcommand{\norme}[1]{\left\Vert #1\right\Vert}

% Notations bb 
\newcommand{\N}{\mathbb N}
\newcommand{\R}{\mathbb R}
\newcommand{\Z}{\mathbb Z}
\newcommand{\C}{\mathbb{C}}
\newcommand{\E}{\mathbb{E}}
\newcommand{\1}{\mathbbm 1}
\newcommand{\PP}{\mathbb{P}}
 
\newcommand{\bl}{\{} 
\newcommand{\br}{\}} 

\def\bx{{\bf x}}\def\bx{{\bf x}}

\def\bA{{\bf A}}
\def\bY{{\bf Y}}
\def\bV{{\bf V}}
\def\bQ{{\bf Q}}
\def\ba{{\bf a}}
\def\bc{{\bf c}}
\def\bd{{\bf d}}
\def\bm{{\bf m}}
\def\bg{{\bf g}}
\def\bp{{\bf p}}
\def\bv{{\bf v}}
\def\bx{{\bf x}}
\def\by{{\bf y}}
\def\brho{{\boldsymbol \rho}}
\def\bDelta{{\boldsymbol \Delta}}
\def\bdelta{{\boldsymbol \delta}}

\def\bOmega{{\bf \Omega}}
\def\bbT{{\bf \mathbb{T}}}

\newtheorem{assumption}{Assumption}

\newcommand{\semitransp}[2][35]{\color{fg!#1}#2}

\begin{document}
 

 
\makeFirstFrame

\frame{
  \frametitle{Plan}
  \tableofcontents
}

\setbeamertemplate{blocks}[rounded][shadow=true]
\section{Internship Environment}

%%%%%%%%%%%%%%%%%%%%%%%%%%%%%%%




%%%%%%%%%%%%%%%%%%%%%%%%%%%%%%%

\subsection{Goals and objectives}
\begin{frame}{Goals and objectives}

\only<1>{
Net Revenu for Video Delivery Applications in USA

\begin{center}
\includegraphics[scale=0.28]{figures/amazon.png}
\end{center}
}

\only<2>{
In 2016, More than 96 \% of internet traffic is content.

Video $\longrightarrow$ 60\%

File sharing $\longrightarrow$ 20\%

Web $\longrightarrow$ 20\%

\begin{center}
\includegraphics[scale=0.28]{figures/stat.png}
\end{center}

}

\only<3>{
Mobile vs PC Internet Traffic user $\longrightarrow$ 5G mobile networks


\begin{center}
\includegraphics[scale=0.48]{figures/mobile.jpg}
\end{center}

}

\end{frame}


\subsection{CISCO \& PIRL}

\begin{frame}{CISCO \& PIRL}

Cisco Systems France.

\begin{center}
\includegraphics[scale=0.07]{figures/cisco.jpg}
\end{center}


\end{frame}



%%%%%%%%%%%%%%%%%%%%%%%
\section{Ideas and Strategies}

\subsection{ICN Brief Introduction}

<<<<<<< HEAD
\begin{frame}{Named Data networking (NDN)}

=======
\begin{frame}{Named Data Networking (NDN)}
\only<1>{
>>>>>>> abb2c5ff0746a5d90f8e6ae6208c9d2c115a06cd
Why ICN?

\begin{itemize}
\item V.Jacobson et al proposition, \textit{Networking Named Content} 2009.
\item \textbf{N}amed \textbf{D}ata \textbf{N}etworking $\Rightarrow$ \textit{\textbf{Name}} base Philosophy vs TCP/IP \textit{\textbf{Calling}} Networking.
\item A Good fit network designing for Video Delivery Applications in \textbf{5G}.

\end{itemize}

\end{frame}


\subsection{Virtualization and Linux Containers}



\begin{frame}{Virtualization and Linux Containers}
\only<1>{
Virtual Machines (VM) vs Linux Containers.

\begin{center}
\includegraphics[scale=0.42]{figures/lxc.png} 
\end{center}

}

\end{frame}

\begin{frame}{Lurch}

<<<<<<< HEAD
\begin{itemize}
\item \textbf{Lurch} is an orchestrator originally developped for ccnx. 
\item We developped Lurch:

\begin{itemize}
\item For NFD (NDN forwarder).
\item Different interfaces to interact with strategies at run time (Client, Repositories, forwading strategies, ...)
\item \textit{New Routing Strategies}.



\begin{center}
\includegraphics[scale=0.25]{figures/controller.png}
\end{center}


\end{itemize}
\end{itemize} 



=======
>>>>>>> abb2c5ff0746a5d90f8e6ae6208c9d2c115a06cd

\end{frame}


\begin{frame}{Large Scale Platform Grid5000}

<<<<<<< HEAD
Grid5000 platform
=======


\subsection{Virtualization and Linux Containers}



\begin{frame}{Virtualization and Linux Containers}
\only<1>{
Virtual Machines (VM) vs Linux Containers.
>>>>>>> abb2c5ff0746a5d90f8e6ae6208c9d2c115a06cd

\begin{center}
\includegraphics[scale=0.48]{figures/grid.jpg}
\end{center}
}
\only<2>{

\begin{itemize}
\item \textbf{Lurch} is an orchestrator originally developped for ccnx. 
\item We developped Lurch:

\begin{itemize}
\item For NFD (NDN forwarder).
\item Different interfaces to interact with strategies at run time (Client, Repositories, forwading strategies, ...)
\item \textit{New Routing Strategies}.

\end{itemize}
\end{itemize}

\begin{center}
\includegraphics[scale=0.25]{figures/controller.png}
\end{center}
}

<<<<<<< HEAD
=======
\only<3>{
\begin{center}
\includegraphics[scale=0.15]{figures/architecture.png}
\end{center}
}

>>>>>>> abb2c5ff0746a5d90f8e6ae6208c9d2c115a06cd
\end{frame}

\begin{frame}{Virtualization Stack}


\begin{center}
\includegraphics[scale=0.15]{figures/architecture.png}
\end{center}

\end{frame}



%%%%%%%%%%%%%%%%%%%%%

<<<<<<< HEAD
%%%%%%%%%%%%%%%%%%%%%%%%

\section{Routing Algorithms Results}
=======

\section{Routing Algorithms Results}

\subsection{Routing Strategies}
>>>>>>> abb2c5ff0746a5d90f8e6ae6208c9d2c115a06cd
\begin{frame}{Routing Strategies}
We proposed 4 different routing strategies for different situation of networks which can cover all of needs:
\begin{itemize}

\item \textbf{TreeOnConsumer} : N clients searching the same content from one repository detected by Lurch (Multicast mode).

\item \textbf{TreeOnProducer}: One client who gets the packet from N Repositories of needed data.

\item \textbf{MinCostMultiPath}: Using different paths with Equal Cost to retrieve the data using a proper forwarder strategy (load-balancing).

\item \textbf{MaxFlow}: Allow to maximize the throughput using paths based on maximum flow algorithm between clients and repositories.

\end{itemize}
\end{frame}
<<<<<<< HEAD
=======
%%%%%%%%%%%%%%%%%%%%%%%%
>>>>>>> abb2c5ff0746a5d90f8e6ae6208c9d2c115a06cd

\subsection{TreeOnConsumer}

\begin{frame}{TreeOnConsumer}

\only<1>{
One producer to multiple consumer.

\includegraphics[scale=0.32]{figures/TreeOnConsumer.png} 
\includegraphics[scale=0.22]{figures/treeonconsumer.png} 
}
\only<2>{
One producer to multiple consumer.

\includegraphics[scale=0.22]{figures/TreeOnConsumer_big.png} 
\includegraphics[scale=0.23]{figures/treeonconsumer_big.png} 

}

\end{frame}

%%%%%%%%%%%%%%%
\subsection{TreeOnProducer}

\begin{frame}{TreeOnProducer}
\only<1>{
One Consumer to multiple producer.

\includegraphics[scale=0.32]{figures/TreeOnProducer.png} 
\includegraphics[scale=0.22]{figures/treeonproducer.png} 
}
\only<2>{
One Consumer to multiple producer.

\includegraphics[scale=0.22]{figures/TreeOnProducer_big.png} 
\includegraphics[scale=0.21]{figures/TreeOnProducer_big.pdf} 

}



\end{frame}


\subsection{MinCostMultiPath}

\begin{frame}{MinCostMultiPath}


\only<1>{
Load balancing strategies in equal cost multipath case.

\includegraphics[scale=0.22]{figures/Load.png} 
\includegraphics[scale=0.25]{figures/load.pdf} 
}
\only<2>{
\includegraphics[scale=0.22]{figures/MinCostMultipath_big.png} 
\includegraphics[scale=0.21]{figures/mincostmultipath_big.png} 

}


\only<3>{

Producer Mobility with Routing update.

\includegraphics[scale=0.20]{figures/Step1.png} 
\includegraphics[scale=0.20]{figures/Step2.png} 


}



\end{frame}




\subsection{Maximum Flow}

\begin{frame}{Maximum Flow}


\only<1>{
Maximum Flow algorithm chooses the path which maximizes through from consumer to producer.

\includegraphics[scale=0.32]{figures/MaxFlow.png} 
\includegraphics[scale=0.22]{figures/maxflow.png} 
}
\only<2>{
Maximum Flow algorithm chooses the path which maximizes through from consumer to producer.

\begin{center}
\includegraphics[scale=0.32]{figures/MaxFlow2.png} 
\end{center}



}




\end{frame}

\section{Conclusion}

\begin{frame}{Conclusion}
\begin{itemize}
 
\item There is always some limitations in practical against pure theoritical works which can be seen when you work on experimental platforms.
\item ICN is one of the most challenging domain who has a lot of field of work and domain in research and development.
\item In Engineering there is always bottlnecks, understanding and discovering of where these bottlenecks are the responsbility of genius engineer.

\item Coding is beautiful tool because through it you can realize your ideas in real world.
\end{itemize}

\end{frame}

%%%%%%%%%%%%%%%%%%%%%%%

\begin{frame}



\begin{figure}[h!]
  \centering
    \includegraphics[scale=0.5]{figures/merci.jpg}
\end{figure}



\end{frame}



\end{document}
